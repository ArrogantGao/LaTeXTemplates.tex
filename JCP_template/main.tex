%% ****** Start of file aiptemplate.tex ****** %
%%
%%   This file is part of the files in the distribution of AIP substyles for REVTeX4.
%%   Version 4.1 of 9 October 2009.
%%
%
% This is a template for producing documents for use with 
% the REVTEX 4.1 document class and the AIP substyles.
% 
% Copy this file to another name and then work on that file.
% That way, you always have this original template file to use.

\documentclass[aip,reprint]{revtex4-1}
%\documentclass[aip,reprint]{revtex4-1}

\usepackage{amsmath,amsthm,amssymb}
\usepackage{graphicx}% Include figure files

\usepackage{dcolumn}% Align table columns on decimal point
\usepackage{bm}% bold math
\usepackage{color}
\usepackage{epsfig}
\usepackage{multirow}
\usepackage{mathrsfs}
\usepackage{hyperref}
\usepackage{cleveref}
\usepackage{epstopdf}
\usepackage{subfigure}
\usepackage{autobreak}

\newcommand{\V}[1]{\boldsymbol{#1}} %# vector
\newcommand{\M}[1]{\boldsymbol{#1}} %# matrix
\newcommand{\Set}[1]{\mathbb{#1}} %# set
\newcommand{\D}[1]{\Delta#1} %# \D{t} for time step size
\renewcommand{\d}[1]{\delta#1} %# \d{t} for small increment
\newcommand{\norm}[1]{\left\Vert #1\right\Vert } % norm
\newcommand{\abs}[1]{\left|#1\right|} %abs

\newcommand{\grad}{\M{\nabla}} %gradient
\newcommand{\av}[1]{\left\langle #1\right\rangle } %take average

\newcommand{\sM}[1]{\M{\mathcal{#1}}} %matrix in mathcal font or operator
\newcommand{\dprime}{\prime\prime} % double prime
%\global\long\def\i{\iota}
%\renewcommand{\i}{\iota} %i for imaginary unit
%\renewcommand{\i}{\mathsf i} %i for imaginary unit
\newcommand{\follows}{\quad\Rightarrow\quad} %=>
\newcommand{\eqd}{\overset{d}{=}} %=^d
\newcommand{\spe}[1]{\mathscr{#1}}  %important quantities in mathscr font
\newcommand{\eps}{\epsilon}
\usepackage{color}
\newcommand{\rev}[1] {{\color{blue}#1}}
\newcommand{\gao}[1] {{\color{red}{\textit{xz comment}}: \color{red}#1}}


\usepackage[utf8]{inputenc}
\usepackage[T1]{fontenc}
\usepackage{mathptmx}
\usepackage{etoolbox}

%% Apr 2021: AIP requests that the corresponding 
%% email to be moved after the affiliations
%\makeatletter
%\def\@email#1#2{%
% \endgroup
% \patchcmd{\titleblock@produce}
%  {\frontmatter@RRAPformat}
%  {\frontmatter@RRAPformat{\produce@RRAP{*#1\href{mailto:#2}{#2}}}\frontmatter@RRAPformat}
%  {}{}
%}%

%\draft
\makeatother
\begin{document}

% Use the \preprint command to place your local institutional report number 
% on the title page in preprint mode.
% Multiple \preprint commands are allowed.
%\preprint{}

\title{Title} %Title of paper

% repeat the \author .. \affiliation  etc. as needed
% \email, \thanks, \homepage, \altaffiliation all apply to the current author.
% Explanatory text should go in the []'s, 
% actual e-mail address or url should go in the {}'s for \email and \homepage.
% Please use the appropriate macro for the type of information

% \affiliation command applies to all authors since the last \affiliation command. 
% The \affiliation command should follow the other information.

\author{Author}
\email[]{email}
\affiliation{affiliation}


% Collaboration name, if desired (requires use of superscriptaddress option in \documentclass). 
% \noaffiliation is required (may also be used with the \author command).
%\collaboration{}
%\noaffiliation

\date{\today}

\begin{abstract}
    This is an abstract.
\end{abstract}

\pacs{}% insert suggested PACS numbers in braces on next line

\maketitle %\maketitle must follow title, authors, abstract and \pacs
%\section{Introduction}

\section{Introduction}

This is an intro.



\bibliography{groupbib}

\end{document}
%
% ****** End of file aiptemplate.tex ******